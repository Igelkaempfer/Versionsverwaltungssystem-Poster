% !TEX TS-program = pdflatex
% !TEX encoding = UTF-8 Unicode

% This is a simple template for a LaTeX document using the "article" class.
% See "book", "report", "letter" for other types of document.

\documentclass[12pt]{article} % use larger type; default would be 10pt
\RequirePackage{multicol}
\columnsep=30mm



\usepackage[utf8]{inputenc} % set input encoding (not needed with XeLaTeX)
\usepackage[T1]{fontenc}
\usepackage{lmodern}


\renewcommand{\Huge}{\fontsize{78}{46}\selectfont}
\renewcommand{\huge}{\fontsize{48}{48}\selectfont}
\renewcommand{\Large}{\fontsize{30}{30}\selectfont}
\renewcommand{\large}{\fontsize{22}{22}\selectfont}
%%% Examples of Article customizations
% These packages are optional, depending whether you want the features they provide.
% See the LaTeX Companion or other references for full information.

%%% PAGE DIMENSIONS
\usepackage{geometry} % to change the page dimensions
\geometry{a1paper} % or letterpaper (US) or a5paper or....
\geometry{margin=2in} % for example, change the margins to 2 inches all round
\geometry{voffset=-4cm} % for example, change the margins to 2 inches all round
% \geometry{landscape} % set up the page for landscape
%   read geometry.pdf for detailed page layout information

\usepackage{graphicx} % support the \includegraphics command and options
\usepackage[font={large,sf}]{caption}
\renewcommand{\figurename}{Abb.}
% \usepackage[parfill]{parskip} % Activate to begin paragraphs with an empty line rather than an indent

%%% PACKAGES
\usepackage{booktabs} % for much better looking tables
\usepackage{array} % for better arrays (eg matrices) in maths
\usepackage{paralist} % very flexible & customisable lists (eg. enumerate/itemize, etc.)
\usepackage{verbatim} % adds environment for commenting out blocks of text & for better verbatim
\usepackage{subfig} % make it possible to include more than one captioned figure/table in a single float
% These packages are all incorporated in the memoir class to one degree or another...

%%% HEADERS & FOOTERS
\usepackage{fancyhdr} % This should be set AFTER setting up the page geometry
\pagestyle{fancy} % options: empty , plain , fancy
\renewcommand{\headrulewidth}{0pt} % customise the layout...
\lhead{}\chead{}\rhead{}
\lfoot{Hallo}\cfoot{\thepage}\rfoot{}
\renewcommand{\footrulewidth}{2pt}

%%% SECTION TITLE APPEARANCE
\usepackage{sectsty}
\allsectionsfont{\sffamily\mdseries\upshape} % (See the fntguide.pdf for font help)
% (This matches ConTeXt defaults)

%%%%%%%%%%%%%%%%%


%%% ToC (table of contents) APPEARANCE
\usepackage[nottoc,notlof,notlot]{tocbibind} % Put the bibliography in the ToC
\usepackage[titles,subfigure]{tocloft} % Alter the style of the Table of Contents
\renewcommand{\cftsecfont}{\rmfamily\mdseries\upshape}
\renewcommand{\cftsecpagefont}{\rmfamily\mdseries\upshape} % No bold!

%%% END Article customizations

%%% The "real" document content comes below...

%-- einige THB-Poster-Sachen
\RequirePackage{color}%
\definecolor{fhborange}{cmyk}{0,0.5,1.0,0}
\definecolor{fhbwhite}{rgb}{1.0,1.0,1.0}




\begin{document}
\sf
\begin{minipage}[b]{12cm}
    \vspace{0mm}	
    	\hspace{-50mm}	
      \includegraphics[height=83mm]{2015_10_05_THB_FB-W_Logo_RGB}
    %\epsfig{file=FHB_Logo.eps,height=6cm}
\end{minipage}
\begin{minipage}{480mm}
    \vspace{-30mm}	 
    	\hspace{-20mm}	
    \color{fhborange}\rule{550mm}{35mm}
\end{minipage}

\vspace{-5mm}

 \begin{minipage}[b]{450mm}
     \hspace{-30mm}	
     \vspace{-25mm}
      \color{fhborange}\rule{580mm}{90mm}
     \vspace{-60mm}	 \\
     	
      \color{fhbwhite}{\Huge 
Projektstudium 2022
\\ \huge 
Projektbetreuer*innen: \\Michael Höding, Kai Jander, Olga Levina, André Nitze
}
\end{minipage}
 
\vspace{4cm}	
\begin{multicols}{3}
\large
\sf

%%Hier startet der Posterinhalt
\section{Das Projektstudium in der WI, dem tollsten Jahrgang aller Zeiten}
Das Projektstudium Wirtschaftsinformatik wird im ersten Studienjahr angeboten. Es dient sowohl der Vermittlung fachlicher als auch überfachlicher Kompetenzen. 

Zum einen werden die Techniken des wissenschaftlichen Arbeitens vertieft. 
Hierbei werden sehr unterschiedlichen Themen aus dem weiten Spektrum der Wirtschaftsinformatik bearbeitet. 


Zum einen werden die Techniken des wissenschaftlichen Arbeitens vertieft. 
Hierbei werden sehr unterschiedlichen Themen aus dem weiten Spektrum der Wirtschaftsinformatik bearbeitet. 

Zum einen werden die Techniken des wissenschaftlichen Arbeitens vertieft. 
Hierbei werden sehr unterschiedlichen Themen aus dem weiten Spektrum der Wirtschaftsinformatik bearbeitet. 

Zum anderen werden durch Gruppenarbeit Fähigkeiten wie Kooperation und Kommunikation, Zeitmanagement und  Projektorganisation entwickelt. 

Im Projektstudium sind mehrere Professoren und Mitarbeiter des WI-Kollegiums als Auftraggeber und Betreuer aktiv und ermöglichen so einen integrierten Einstieg in das Studium.

Im Rahmen der Abschlussveranstaltung wird der wissenschaftliche Austausch gepflegt.

%
Das Projektstudium dient der Übung und Vertiefung zum Themengebiet wissenschaftliches Arbeitem. Besonderheiten sind 
\begin{itemize}
\item Werkstattarbeit in Projektgruppe, 
\item Abschlußpräsentationen im Großen Hörsaal, 
\item Poster-Ausstellung
\end{itemize}
Ein weiteres Ergebnis soll die Erstellung einer gemeinsamen Bibliography mit Citavi sein.

%
%
%\noindent\begin{minipage}[h]{14cm}
% \centering
%   \includegraphics[width=.8\textwidth]{lernkurve.png}
%\end{minipage}
%\begin{center}{A graphic that needs to span two columns of text.}\end{center}
%



\vspace{1mm}
\noindent
\begin{minipage}[h]{14cm}
 \centering
   \includegraphics[width=14cm]{FotoLevina}
\captionof{figure}{Projekt WOM}
\end{minipage}
Daher wollen wir uns mit der Frage auseinandersetzten wie sinnvoll ein Community-Feedback-System für die THB Campus APP ist und welche Verbesserungen man implementieren könnte. Dieses Ziel errei\-chen wir indem wir die bestehende APP analysieren und Umfragen bezüglich Bewertung und Verbesserungvorschläge erstellen. 




\begin{thebibliography}{99}
\bibitem{SZG}
A.~Picot, S.~Zahedani, A.~Ziemer: Spielend die Zukunft gewinnen (2008)
\bibitem{stat}
Bundesverband Interaktive Unterhaltungssoftware e. V.: biu-online.de
\bibitem{SK}
Sparkassen Planspiel Börse: www.planspiel-boerse.de
\bibitem{FHB}
Modulkatalog des Studiengang Wirtschaftsinformatik:\\
http://fbwcms.fh-brandenburg.de/de/6230
\end{thebibliography}


\end{multicols} 

\lfoot{
\Large Dieses Poster ist im Rahmen der Lehrveranstaltung \emph{Projektstudium -
Wissenschaftliches Arbeiten} im Sommersemester 2022 entstanden.
\vspace{1cm}\\
\begin{minipage}[b]{20 cm} 
\sffamily\fontsize{30}{30}\selectfont
Ihr Kontakt: Prof. Dr. Michael Höding \\
TH Brandenburg, Postfach 2132\\
D-14737 Brandenburg, Germany\\E-Mail: \texttt{hoeding@th-brandenburg.de}
\end{minipage}
\begin{minipage}[b]{13 cm}
.
\end{minipage}
\begin{minipage}[b]{15 cm}
\sffamily\fontsize{20}{10}\selectfont
.der zweite Punkt
\end{minipage}
}
\end{document}
