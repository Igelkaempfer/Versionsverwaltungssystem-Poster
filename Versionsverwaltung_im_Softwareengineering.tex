% !TEX TS-program = pdflatex
% !TEX encoding = UTF-8 Unicode

% This is a simple template for a LaTeX document using the "article" class.
% See "book", "report", "letter" for other types of document.

\documentclass[12pt]{article} % use larger type; default would be 10pt
\RequirePackage{multicol}
\columnsep=30mm



\usepackage[utf8]{inputenc} % set input encoding (not needed with XeLaTeX)
\usepackage[T1]{fontenc}
\usepackage{lmodern}


\renewcommand{\Huge}{\fontsize{78}{46}\selectfont}
\renewcommand{\huge}{\fontsize{48}{48}\selectfont}
\renewcommand{\Large}{\fontsize{30}{30}\selectfont}
\renewcommand{\large}{\fontsize{22}{22}\selectfont}
%%% Examples of Article customizations
% These packages are optional, depending whether you want the features they provide.
% See the LaTeX Companion or other references for full information.

%%% PAGE DIMENSIONS
\usepackage{geometry} % to change the page dimensions
\geometry{a1paper} % or letterpaper (US) or a5paper or....
\geometry{margin=2in} % for example, change the margins to 2 inches all round
\geometry{voffset=-4cm} % for example, change the margins to 2 inches all round
% \geometry{landscape} % set up the page for landscape
%   read geometry.pdf for detailed page layout information

\usepackage{graphicx} % support the \includegraphics command and options
\usepackage[font={large,sf}]{caption}
\renewcommand{\figurename}{Abb.}
% \usepackage[parfill]{parskip} % Activate to begin paragraphs with an empty line rather than an indent

%%% PACKAGES
\usepackage{booktabs} % for much better looking tables
\usepackage{array} % for better arrays (eg matrices) in maths
\usepackage{paralist} % very flexible & customisable lists (eg. enumerate/itemize, etc.)
\usepackage{verbatim} % adds environment for commenting out blocks of text & for better verbatim
\usepackage{subfig} % make it possible to include more than one captioned figure/table in a single float
% These packages are all incorporated in the memoir class to one degree or another...

%%% HEADERS & FOOTERS
\usepackage{fancyhdr} % This should be set AFTER setting up the page geometry
\pagestyle{fancy} % options: empty , plain , fancy
\renewcommand{\headrulewidth}{0pt} % customise the layout...
\lhead{}\chead{}\rhead{}
\lfoot{Hallo}\cfoot{\thepage}\rfoot{}
\renewcommand{\footrulewidth}{2pt}

%%% SECTION TITLE APPEARANCE
\usepackage{sectsty}
\allsectionsfont{\sffamily\mdseries\upshape} % (See the fntguide.pdf for font help)
% (This matches ConTeXt defaults)

%%%%%%%%%%%%%%%%%


%%% ToC (table of contents) APPEARANCE
\usepackage[nottoc,notlof,notlot]{tocbibind} % Put the bibliography in the ToC
\usepackage[titles,subfigure]{tocloft} % Alter the style of the Table of Contents
\renewcommand{\cftsecfont}{\rmfamily\mdseries\upshape}
\renewcommand{\cftsecpagefont}{\rmfamily\mdseries\upshape} % No bold!

%%% END Article customizations

%%% The "real" document content comes below...

%-- einige THB-Poster-Sachen
\RequirePackage{color}%
\definecolor{fhborange}{cmyk}{0,0.5,1.0,0}
\definecolor{fhbwhite}{rgb}{1.0,1.0,1.0}




\begin{document}
\sf
\begin{minipage}[b]{12cm}
    \vspace{0mm}	
    	\hspace{-50mm}	
      \includegraphics[height=83mm]{2015_10_05_THB_FB-W_Logo_RGB}
    %\epsfig{file=FHB_Logo.eps,height=6cm}
\end{minipage}
\begin{minipage}{480mm}
    \vspace{-30mm}	 
    	\hspace{-20mm}	
    \color{fhborange}\rule{550mm}{35mm}
\end{minipage}

\vspace{-5mm}

 \begin{minipage}[b]{450mm}
     \hspace{-20mm}	
     \vspace{-25mm}
      \color{fhborange}\rule{580mm}{90mm}
     \vspace{-70mm}	 \\
     	
      \color{fhbwhite}{\Huge 
Versionsverwaltung im Softwareengineering
\\ \huge 
Projektbetreuer*innen: \\Michael Höding, Ralf Teusner
}
\end{minipage}
 
\vspace{4cm}	
\begingroup
\sf
\large
\begin{minipage}[t]{0.32\textwidth}
%% Inhalt der ersten Spalte hier einfügen
\section{Das Projektstudium in 
der WI, dem tollsten Jahrgang aller Zeiten}
Das Projektstudium Wirtschaftsinformatik wird im ersten Studienjahr angeboten.
Es dient sowohl der Vermittlung fachlicher als auch überfachlicher Kompetenzen. 
% ... (Weitere Inhalte, die nur in die erste Spalte passen sollen) ...
Zum anderen werden durch Gruppenarbeit Fähigkeiten wie Kooperation und Kommunikation, Zeitmanagement und  Projektorganisation entwickelt.
\end{minipage}%
\hspace{0.02\textwidth}% Ein kleiner horizontaler Abstand zwischen den Spalten (2% der Textbreite)
\begin{minipage}[t]{0.32\textwidth}

%Platz für erste LeitFrage
\section{Was sind Branching-Strategien?}
Branching-Strategien definieren strukturierte Vorgehensweisen zur Erstellung, Pflege und Zusammenführung paralleler Entwicklungszweige in Versionsverwaltungssystemen.
Sie steuern, wie Teams neue Features entwickeln, Fehler beheben oder Releases vorbereiten, ohne Code-Stabilität zu gefährden.
Sauber definierte Branch-Modelle reduzieren Konflikte, beschleunigen Reviews und erhöhen Nachvollziehbarkeit.
Typische Ansätze sind Feature Branching, GitFlow, Trunk-Based Development und Release-Branches.

% ... (Der zweite Teil des Inhalts) ...
\section{Gitflow}
Git Flow ist ein robustes Branching-Modell, das auf Git basiert und einen strikten Rahmen für die Verwaltung von Projektreleases bietet.
Es definiert fünf Haupt-Branch-Typen mit spezifischen Aufgaben und Lebenszyklen: 
\begin{itemize}
\item main (oder master): Enthält den produktionsreifen Code. 
\item develop: Integriert die neuesten fertigen Features für den nächsten Release. 
\item feature Branches: Dienen der Entwicklung neuer Funktionalitäten und basieren auf develop. 
\item release Branches: Werden zur Vorbereitung eines neuen Releases von develop erstellt (Bugfixes, Metadaten-Updates)
\item hotfix Branches: Werden direkt von main erstellt, um kritische Fehler in der Produktion schnell zu beheben
\end{itemize}
\section{Trunk-Based Development}
Trunk-Based Development ist ein agiler Ansatz, bei dem alle Entwickler kontinuierlich in einen gemeinsamen Hauptzweig
(„Trunk“) integrieren. Änderungen werden in kleinen, häufigen Commits vorgenommen, wodurch Merge-Konflikte reduziert
und die Software stets in einem lauffähigen Zustand gehalten wird. Feature-Flags ermöglichen das schrittweise Aktivieren neuer Funktionen.
Dieser Ansatz fördert schnelle Feedbackzyklen, höhere Codequalität und eine stabilere Release-Pipeline.
\end{minipage}%
\hspace{0.02\textwidth}% Ein kleiner horizontaler Abstand zwischen den Spalten (2% der Textbreite)
\begin{minipage}[t]{0.32\textwidth}
%% Inhalt der dritten Spalte hier einfügen
\section{Feature Branching}
Feature Branching ist eine gängige Strategie in der Versionskontrolle (z.B. Git), 
bei der Entwickler für jede neue Funktion, Fehlerbehebung oder jedes Experiment einen separaten Branch von der Hauptentwicklungslinie erstellen. 
Dadurch wird die Isolation des Codes gewährleistet. Die Arbeit kann unabhängig und ohne Risiko, die stabile Hauptversion zu beeinträchtigen, durchgeführt werden.
Erst nach Fertigstellung und erfolgreicher Überprüfung wird der Feature Branch wieder in den Main Branch zusammengeführt.
\section{Forking Workflow}
Der Forking-Workflow unterscheidet sich fundamental von anderen Arbeitsweisen, da er keine Schreibrechte für alle Entwickler auf einem zentralen Repository voraussetzt.
Stattdessen erstellt jeder Beteiligte eine serverseitige Kopie (Fork) des Projekts. Dieses Modell hat sich als De-facto-Standard in der Open-Source-Entwicklung etabliert,
da es eine sichere Zusammenarbeit ohne zentrale Verwaltung von Zugriffsrechten ermöglicht. 
Dabei fungieren eine oder mehrere vertrauenswürdige Personen (Maintainer) als „Gatekeeper“. Nur sie besitzen die Berechtigung,
Änderungen in das offizielle Hauptrepository zu integrieren, wodurch eine klare Hierarchie und Qualitätssicherung gewährleistet wird.
\noindent
\begin{minipage}[h]{\textwidth} % Beachten Sie, dass \textwidth hier 0.32\textwidth entspricht
 \centering
   \includegraphics[width=\textwidth]{github1}
\end{minipage}
\section{Wie beeinflussen unterschiedliche Branching-Strategien die Softwarequalität, insbesondere im Hinblick auf Fehlerhäufigkeit, Code-Stabilität und Merge-Konflikte?}
\noindent
\begin{minipage}[h]{\textwidth} % Beachten Sie, dass \textwidth hier 0.32\textwidth entspricht
 \centering
   \includegraphics[width=\textwidth]{Tabelle}
\end{minipage}
\end{minipage}
% ... (Der dritte Teil des Inhalts) ...
\endgroup

\lfoot{
\Large Dieses Poster ist im Rahmen der Lehrveranstaltung \emph{Projektstudium -
Wissenschaftliches Arbeiten} im Sommersemester 2022 entstanden.
\vspace{1cm}\\
\begin{minipage}[b]{20 cm} 
\sffamily\fontsize{30}{30}\selectfont
Ihr Kontakt: Prof. Dr. Michael Höding \\
TH Brandenburg, Postfach 2132\\
D-14737 Brandenburg, Germany\\E-Mail: \texttt{hoeding@th-brandenburg.de}
\end{minipage}
\begin{minipage}[b]{13 cm}
.
\end{minipage}
\begin{minipage}[b]{15 cm}
\sffamily\fontsize{20}{10}\selectfont
.der zweite Punkt
\end{minipage}
}
\end{document}
